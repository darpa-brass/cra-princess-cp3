% common settings
% Common settings for latex'd docs

% layout
\documentclass[10pt, twocolumn, letterpaper]{article}
\usepackage{fullpage}
\frenchspacing
%\usepackage{parskip}

% fonts
\usepackage{charter}
\usepackage{courier}

% other stuff
\usepackage{graphicx}
\usepackage{url}
%\usepackage{mathpartir}
\usepackage{amsmath}
\usepackage{amssymb}
\usepackage{wasysym}

% without help, latex sets these characters in the wrong font inside \tt
\newcommand{\us}{{\tt\char`\_}}
\newcommand{\bra}{{\tt\char`\{}}
\newcommand{\ket}{{\tt\char`\}}}
\newcommand{\bs}{{\tt\char`\\}}

\usepackage{color}
\newcommand{\XXX}[1]{{\bf\color{red} #1}}

%\usepackage{amsthm}
%\theoremstyle{plain}
%\newtheorem{theorem}{Theorem}
%\theoremstyle{definition}
%\newtheorem{definition}{Definition}
%\newtheorem{proposition}{Proposition}
%\newtheorem*{example}{Example}

% I hate latex
%\newcommand{\mathsc}[1]{{\normalfont\textsc{#1}}}

\usepackage{ascii}
\newcommand{\altmid}{\mathop{\textrm{\splitvert}}}

\newcommand{\bintools}{\texttt{bintools}}



\title{Retargeting \bintools}
\author{David A. Holland}
\date{April 2, 2018}

\begin{document}
\maketitle
%%%%%%%%%%%%%%%%%%%%%%%%%%%%%%%%%%%%%%%%%%%%%%%%%%%%%%%%%%%%

\section{Introduction}

\textit{
This document is about adapting {\bintools} to a new processor
architecture.
For information about adding support for a new binary format, see
the file \texttt{binfmts.pdf}.
For information about adding support for a new operating system using
an existing processor architecture and binary format, see the file
\texttt{porting.pdf}.
}

The {\bintools} suite is retargetable; this means it supports multiple
processor architectures, like a retargetable compiler.
This is accomplished via \emph{machine descriptions}: relatively
concise and non-Turing-complete specifications of processor
architectures that are used to generate the needed code.

Porting {\bintools} to a new processor architecture requires writing a
machine description; it does not require writing code, and, with luck,
should also not entail extending the machine description framework to
represent the new machine.

The machine descriptions used by {\bintools} are found in the
\texttt{targets/} directory of its source tree.
Each description is itself a directory, containing a number files that
describe particular aspects of the machine architecture.

These files are:
\begin{itemize}
\item \texttt{base.def} - Core types used by the other description files
\item \texttt{misc.def} - Assorted fundamental machine properties and
miscellaneous configuration
\item \texttt{modes.def} - Modes that exist in the assembler 
\item \texttt{modifiers.def} - ``Modifiers'', machine-dependent
operators appearing in the assembly language
\item \texttt{language.def} - Syntax and other properties of the
target machine's assembly language
\item \texttt{opmatch.def} - Table matching assembly-language-level
opcodes to machine-level instructions and encodings
\item \texttt{encoding.def} - Definitions of instruction encodings
\item \texttt{hazards.def} - Specification of branch instructions and
pipeline hazards for hazard checking
\item \texttt{relocations.def} - Definition of linker relocations
\end{itemize}

The full specification for each of these files may be found below;
first, however, we need to define some shared concepts used by the
description system.

\section{Concepts}

\subsection{Types}

The {\bintools} machine description system is strongly typed.
\XXX{...}

booleans, strings

integer types

enumerated types

subenums

compound types

type inference

\subsection{Endianness and Bit Size}
\subsection{PC Unit}
\subsection{Operand Tags}
\subsection{Instruction Fields and Forms}
\subsection{Machine Level vs. Language Level Instructions}
\subsection{Synthetic Instructions/Macros}
\subsection{Branches and Jumps}
\subsection{Pipeline Hazards}
\subsection{Register Classes}
\subsection{Assembler Modes}
\subsection{Alignment}
\subsection{Modifiers}


\section{Writing a New Description}

There are roughly four major parts to a new machine description, once
the basic machine properties are given.
These are: defining the assembly language syntax, defining the
language-level instructions and directives, defining the bit-level
instructions, and defining linker relocations and linker
configuration.

These parts correspond to the workflow of assembling and linking:
given some assembly language (e.g. emitted by a compiler), first the
assembler parses it into a sequence of instructions and directives;
then the assembler turns the instructions into output bits and bytes
and outputs relocations for the linker to resolve.

\subsection{Basic Properties}

The following basic properties go in \texttt{misc.def}:
\begin{itemize}
\item Endianness
\item Integer encoding
\item Byte, address, instruction fetch, and maximum integer bit width
\end{itemize}
The syntax for these properties is \texttt{property: setting}
for enumerated alternatives, and \texttt{property = value} for
integer values.

\paragraph{Endianness.} 
The machine endianness can be \texttt{big} (big-endian, most
significant byte first, network byte order), \texttt{little}
(little-endian, least significant byte first, not network byte order)
or \texttt{dynamic}, meaning the machine supports both endiannesses at
runtime.
Generally if the endianness is fixed (e.g. all x86 processors are
little-endian) it should be given a fixed value here; otherwise make
it dynamic.
Example:
\begin{verbatim}
endianness: big;
\end{verbatim}

% XXX
Restrictions: currently dynamic endianness does not work; currently
there is no way to configure the default for dynamic-endian machines
(this goes in OS config, not machine config); and currently there is
no support for mixed-endian orderings.

\paragraph{Integer encoding.}
The integer encoding can in theory be two's complement, one's
complement, or sign-magnitude.
Note however that all non-archaic hardware is two's complement.
Example:
\begin{verbatim}
intencoding: twos_complement;
\end{verbatim}

Restrictions: while you can configure \texttt{ones{\us}complement} or
\texttt{sign{\us}magnitude}, in practice it is unlikely to work
properly, and many potential problems lurk in trying to cross-assemble
between machines types with different integer encoding.
For the time being, alternate integer encodings should be treated as
unsupported.

\paragraph{Byte width.}
For the purposes of {\bintools} a byte is the minimum addressable unit
of memory.
On all modern hardware the byte width is 8.
For old word-addressed machines the byte with is generally the word
size.
\XXX{Figure out what to recommend for e.g. 36-bit word-addressed
machines with an implicit byte size of 9...}
Example:
\begin{verbatim}
bytewidth = 8;
\end{verbatim}

Restrictions: byte widths less than 8 are unlikely to work.

\paragraph{Address width.}
The bit width of a memory address.
This governs the size of the integer type used to do address
arithmetic in the assembler, the linker, and elsewhere, and
consequently the type of assembler and linker symbols.
For segmented machines other than 16-bit x86, this will usually be the
size of the part of the address that gives the offset into the
segment; the segment number should be separate.
\XXX{Will such machines actually work that way? Good question.}
For machines like 16-bit x86 where the ``segments'' are not actually
separate regions of memory, the address width should be the width
necessary to represent the full segment/offset pair, and extra (not
implemented) hooks will be needed to perform the address arithmetic.
For modern hardware the address with will almost always be either 32
or 64.
Example:
\begin{verbatim}
addresswidth = 32;
\end{verbatim}

% XXX
Restrictions: currently there is no way to make this depend on
assembler mode, which is necessary for the desired handling of
architectures with both 32- and 64-bit versions.
Currently there is also no way to make code and data address different
sizes, which is necessary for some DSP- or GPU-type hardware.
It should work to use a larger size as long as no address arithmetic
relies on integer overflow/wraparound.
The address width need not be a multiple of the byte width, although
typically it will be.

\paragraph{Instruction fetch width.}
The ``pcunitwidth'' is the minimum unit of instruction fetch; every
instruction form must be an integer multiple of this size.
It must in turn be an integer multiple of the byte width.
On most RISC architectures it will be 32, but on others it may be
smaller (or larger) - e.g. on x86 it is 8, and on m68k it is 16.
Example:
\begin{verbatim}
pcunitwidth = 32;
\end{verbatim}

Restrictions: as noted.

\paragraph{Maximum integer width.}
The maximum integer width is the size of the largest integer constants
the assembler is expected to handle.
This determines the base type of integer constant tokens.
On most machines this will be 64.
Example:
\begin{verbatim}
maximumwidth = 64;
\end{verbatim}


\subsection{Defining the Assembly Language}

As noted above, the assembly language parser is responsible for
converting input assembly language text into a list of directives and
instructions, possibly annotated with labels.
Since virtually all assembly languages have the same basic form,
the parser is rather stylized.
It consists of the following stages:
\begin{enumerate}
\item Tokenizing the input, in terms of \emph{primitive tokens}
\item Matching sequences of primitive tokens to make \emph{target tokens}
\item Dividing into lines (one label and/or directive/instruction per line)
\item Dividing each instruction or directive line into an opcode (or
directive name) and operands
\item Parsing and tagging each operand 
\end{enumerate}
Each of these stages can be configured in \texttt{language.def},
except for dividing up lines, which is hardcoded.

\subsubsection{Tokenization}
There are two sets of declarations related to tokenization: one for
comments, and one related to characters.

\paragraph{Comments.}
Declarations about comments go in a block called \texttt{comments}.
There are three possible declarations: \texttt{line}, which is
followed by a quoted string, and declares a string or
character that introduces a single-line comment; \texttt{nest}, which
is followed by two quoted strings, the first the open-comment string
and the second the close-comment string, which define multiline block
comments; and \texttt{flat}, which is the same as \texttt{nest} but
the comments so declared do not nest, like in C.
Example:
\begin{verbatim}
comments {
   line "#";
   nest "/*" "*/";
}
\end{verbatim}

Restrictions: currently the implementation is restricted to only one
multi-character single-line comment string and only the C comment
delimiters are accepted by \texttt{nest} and \texttt{flat}.
This is a bug. % XXX

\paragraph{Characters.}
Declarations about characters go in a block called
\texttt{characters}.
The first declaration gives the character set, which for now must
always be \texttt{ascii}, although \texttt{utf-8} may perhaps appear
in the future.
Further declarations have the form \texttt{letter} followed by a
single-character quoted string, which makes the character so named
into a letter for purposes of lexing identifiers; or
\texttt{subletter}, which is the same except that the character can
only appear within identifiers and not be the first letter.
(All characters declared \texttt{letter} are also implicitly declared
\texttt{subletter}.)
By default underscore (\texttt{\us}) and period (\texttt{.}) are
always letters, but on some platforms other characters must be
accepted as well.
Example:
\begin{verbatim}
comments {
   charset "ascii";
   letter "$";
   subletter "@";
}
\end{verbatim}
\XXX{why isn't the syntax \texttt{charset: ascii} like in base.def?}

Restrictions:
Making letters out of digits or whitespace, or other characters with
fixed meanings like quotes, will break the world and should be
forbidden.
There should be a way to define additional quote characters for
additional kinds of quoted strings.

\subsubsection{Matching}

There are three sections for customizing token matching.
The first, \texttt{tokens}, allows declaring new kinds of tokens;
the others, \texttt{keywords} and \texttt{match}, define rules that
allow producing such tokens.

\paragraph{Tokens.}
The tokens block is a sequence of declarations of \emph{target
tokens}, each being a new variety of input token.
Each declaration is an identifier and an optional argument declaring the value
type that the token holds, if any.
The possible types are:
\begin{enumerate}
\item \texttt{bool}, one of the values \texttt{true} or \texttt{false}
\item \texttt{string}, an arbitrary string value
\item \texttt{number}, an arbitrary integer (of width \texttt{maximumwidth})
\item \texttt{range}, a pair of numbers
\item The name of any enumerated type from \texttt{base.def}
\end{enumerate}
For example, typically there will be at least one register enumeration
in \texttt{base.def} that the parser ultimately needs to be able to
output; thus one will typically declare at least one register token
type.
Token type names are conventioually all upper-case because they
correspond to names of enumerators in the assembler's C code.
Example:
\begin{verbatim}
tokens {
   REGISTER(reg);
   FPUREGISTER(number);
   FPUCTL;
}
\end{verbatim}

Restrictions: none in particular.
\XXX{do tokens without value types work properly?}

\paragraph{Keywords.}
Each entry in a \texttt{keywords} block is a mapping from a specific
string to a token type and value.
The string must be an identifier; when such an identifier is seen in
the input, it is removed and replaced with the token specified on the
right hand side.
This occurs regardless of context.
Example:
\begin{verbatim}
keywords {
   "$0" => REGISTER(zero);
}
\end{verbatim}

% XXX
Restrictions: currently specifying keyword strings that are not
identifiers is not diagnosed \XXX{I think} and will silently fail to
work.
Currently there is no way to match text patterns within keyword
strings; there is also no way to implicitly generate a collection of
keywords by e.g. wrapping an enumeration.

\paragraph{Match rules.}
Each entry in a \texttt{match} block is a mapping from an arbitrary
token sequence to a new token type and value.
The arbitrary token sequence may be written in terms of primitive
tokens or already-defined target tokens.
The values in tokens to be matched (on the left-hand side) may be
literals, in which case the match only occurs if the value of the
token found matches the literal; or they may be symbols, in which case
the symbol is bound when matched and may be used on the right-hand
side to produce a value for the output token.
Example:
\begin{verbatim}
match {
   DOL NUMBER(0) => REGISTER(zero);
   IDENT("$0") => REGISTER(zero);
   PCT IDENT(s) => MODIFIER(s);
}
\end{verbatim}
The first entry is equivalent to the keyword example above, if
\texttt{DOL} (\texttt{\$}) is not declared a letter.
If it \emph{is} declared a letter, the second entry is equivalent to
the keyword example.
The third entry makes any identifier preceded by a percent sign a
``modifier'', and saves the identifier as the value.

Restrictions: it is intended to be able to write quoted strings on the
left-hand side containing text that is unpacked into tokens directly,
but this is not yet implemented.
There is also no way to generate a collection of matching rules by e.g.
wrapping an enumeration.
It should be possible to have guard expressions that test the values
of the tokens on the left-hand side.

\paragraph{Primitive tokens.}
The list of primitive tokens is:
\begin{itemize}
\item \texttt{IDENT} (identifier) whose value type is string
\item \texttt{NUMBER} (numeric constant) whose value type is number
\item \texttt{STRING} (quoted string) whose value type is string
\end{itemize}
as well as the following tokens with no values:
\begin{tabular}{cccccc}
\texttt{BANG} & \texttt{!} &
\texttt{HASH} & \texttt{\#} &
\texttt{DOL} & \texttt{\$} \\
\texttt{PCT} & \texttt{\%} &
\texttt{AMP} & \texttt{\&} &
\texttt{STAR} & \texttt{*} \\
\texttt{PLUS} & \texttt{+} &
\texttt{COMMA} & \texttt{,} &
\texttt{MINUS} & \texttt{-} \\
\texttt{DOT} & \texttt{.} &
\texttt{SLASH} & \texttt{/} &
\texttt{COLON} & \texttt{:} \\
\texttt{SEMIC} & \texttt{;} &
\texttt{LT} & \texttt{<} &
\texttt{LTLT} & \texttt{<<} \\
\texttt{EQ} & \texttt{=} &
\texttt{GT} & \texttt{>} &
\texttt{GTGT} & \texttt{>>} \\
\texttt{QUES} & \texttt{?} &
\texttt{AT} & \texttt{@} &
\texttt{CARET} & \texttt{\^} \\
\texttt{BAR} & \texttt{|} &
\texttt{TILDE} & \texttt{\~} \\
\texttt{LPAREN} & \texttt{(} & \texttt{RPAREN} & \texttt{)} \\
\texttt{LBRACK} & \texttt{[} & \texttt{RBRACK} & \texttt{]} \\
\texttt{LBRACE} & \texttt{\bra} & \texttt{RBRACE} & \texttt{\ket} \\
\end{tabular}

\XXX{Currently \texttt{MODIFIER} is also predefined but this is all
wrong. Probably.}

\XXX{Can you match on \texttt{WS}? What about \texttt{NL}?}

\paragraph{Expressions.}
Simple expressions can be used to produce the values of tokens on the
right hand side of both keyword and match rules.
Currently these are limited to constants, enumeration constants from
enumeration types in \texttt{base.def}, and arithmetic, logical, and
comparison unary and binary operators.
% XXX
It is not intentional that these expressions are less powerful than
the ones elsewhere in the description system.

\paragraph{Additional considerations.}
Note that every keyword declaration can be written as a match rule;
however, the corresponding keyword declarations are shorter.

In some cases for constructs like numbered registers it makes sense to
have a syntactic rule of the form \texttt{PCT NUMBER(n) => REG(n)}, so
any number tagged with a percent sign generates a register token and
the use of an out of range register number (implicitly) produces an
error.
In other cases, especially when the tag character has to be treated as
a letter, it is necessary to generate a specific list of
rules of the form
\texttt{IDENT("\%0") => REG(0)} or
\texttt{PCT NUMBER(0) => REG(0)},
because a percent sign followed by an out-of range register number is
expected to mean something different.
E.g. on MIPS registers are tagged with \texttt{\$} but this is a
letter, and while \texttt{\$31} is a register reference, \texttt{\$32}
is an ordinary identifier and a perfectly legal symbol name.
Currently there is no way to produce 32 specifically-numbered register
match rule besides cutpaste, but this is a bug; a method is planned. % XXX

\subsubsection{Dividing into lines}

Once tokenized, the input is divided into lines at newline tokens.
Some assembly languages allow multiple ``lines'' of assembly, that is,
multiple instructions, on a single line separated by some token, often
semicolon.

This can be declared as follows:
Example:
\begin{verbatim}
syntax {
   separator ";";
}
\end{verbatim}
This is the only declaration allowed in the \texttt{syntax} block.

\subsubsection{Parsing lines}

A line consists of an optional label, which is an identifier or number
followed by a colon, followed by an opcode or directive, followed by
zero or more operands separated by commas.
The parsing for this is hardcoded.
\XXX{doesn't it run \texttt{target{\us}scantokens} on the label? I guess it
actually tokenizes one line at a time currently, and in fact one line
piece at a time. blah... think about this.}


Opcodes are identifiers.
Directives are identifiers preceded by a dot.

\subsubsection{Tagging operands}

\XXX{continue here}


\subsection{Defining Language-Level Instructions}
\subsection{Defining Bit-Level Instructions}
\subsection{Configuring Linking}


Usually the best way to go about this is, for a typical architecture,
is to begin by defining the bit-level instructions and language-level
instructions together, as this material comes directly out of the
architecture manual; then define the assembly language syntax, as


\section{Generating Code From Descriptions}

\section{Machine Description Reference}


%%%%%%%%%%%%%%%%%%%%%%%%%%%%%%%%%%%%%%%%%%%%%%%%%%%%%%%%%%%%
\end{document}
